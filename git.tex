\documentclass{beamer}
\usepackage{beamerthemeshadow}
\usepackage[export]{adjustbox}
\usepackage[utf8]{inputenc}
\usepackage{multicol}

\mode<presentation> {
  \setbeamercovered{transparent}
}

% Code for placing the footnote above the navigiation symbols
\addtobeamertemplate{footnote}{\vspace{-6pt}}{\vspace{6pt}}
\makeatletter
\setbeamerfont{footline}{size=\fontsize{4}{6}\selectfont}
\setbeamerfont{headline}{size=\fontsize{4}{6}\selectfont}
% Alternative A: footnote rule
\renewcommand*{\footnoterule}{\kern -3pt \hrule \@width 2in \kern 8.6pt}
% Alternative B: no footnote rule
% \renewcommand*{\footnoterule}{\kern 6pt}
\makeatother

\begin{document}
\title{Workshop de Git}  
\author{Ing. Rodrigo García}
\institute{Despegar.com}
\date{\today} 

\begin{frame}
\titlepage
\end{frame}

\begin{frame}{Contenido}
\begin{multicols}{2}
\tableofcontents
\end{multicols}
\end{frame}

\section{Introducción} 

\begin{frame}\frametitle{¿Qué es Git?} 
  \begin{block}{Definición}
    Git es un sistema de control de versión libre, de código abierto y \textbf{distribuido}, diseñado
    para manejar desde proyectos chicos a muy grandes en forma veloz y eficiente. \footnotemark
  \end{block} \pause
  
  \begin{block}{Comentario}
    Así como se denomina distribuido, en realidad nosotros lo usamos en forma centralizada. Cuando mandamos nuestros cambios,
    lo hacemos a un repositorio remoto. De todos modos, cada uno tiene en su máquina un repositorio con toda la información
    histórica de las ramas en las que trabajamos, aún cuando no tenemos conexión al servidor, y se puede configurar para usarlo en forma
    distribuida. 
  \end{block}
  \footnotetext{\url{http://git-scm.com/}}
\end{frame}

\section{Instalación}
\subsection{Instalación}
\begin{frame}[fragile]{Instalando...} 
  \begin{block}{}
      \begin{itemize}
      \item Ubuntu (Linux): \$ sudo apt install git \pause
	\item	 Windows: \begin{itemize}
			\item Página oficial: \url{http://git-scm.com/download/win} 
			\item Link alternativo: \url{http://code.google.com/p/msysgit/downloads/list?can=3}
		\item Tortoise Git: \url{http://code.google.com/p/tortoisegit/}
			\item \textbf{Importante:} al instalar, poner la opción putty y no openSSH.
		      \end{itemize} \pause
      \item Mac: \url{http://git-scm.com/download/mac} \pause
      \item GUI para Eclipse: EGit (buscarlo en el marketplace)
      \end{itemize}
  
    Sea cual sea la versión, siempre disponemos de un shell donde ejecutar comandos. En linux y mac es la consola unix,
    y en windows nos instala el Git Shell. 
  \end{block}
  
\end{frame}

\begin{frame}[fragile]{Instalando... (2)}
  \begin{block}{Ayuda!}
    \begin{verbatim}
$ git --help
$ git help <comando>
     \end{verbatim}
  \end{block} \pause
  \begin{block}{En SS.OO. Unix (manual más completo que el help)}
    \begin{verbatim}
$ man git
     \end{verbatim}
  \end{block} \pause
  \begin{block}{En general (man en la web)}
    \url{http://git-scm.com/docs}
  \end{block}
\end{frame}


\subsection{Configuración}
\begin{frame}[fragile]{Configurando...}
  Ejecutamos lo siguiente en el shell de git para configurar nuestro usuario \footnote{Para más información: \url{http://wiki.despegar.it/index.php/Git}}:
  
  \begin{block}{Comandos}
  \begin{verbatim}
$ git config --global user.name "Juan Perez"
$ git config --global user.email "jperez@despegar.com"
  \end{verbatim}
  \end{block}

  Esta información va a servir para indicarle al repositorio quiénes somos y cómo mostrar nuestros commits.
  
\end{frame}

\begin{frame}\frametitle{Configurando... (2)}
  \begin{block}{Generación de claves (Unix)}
  \begin{itemize}
    \item \$ mkdir -p  $\sim$/.ssh/ \pause 
    \item \$ cd $\sim$/.ssh/ \pause 
    \item \$ ssh-keygen -C jperez@despegar.com -f id\_desp\_rsa \pause 
    \item Ir a \url{https://github.com/settings/keys} y agregar la clave generada en id\_desp\_rsa.pub
  \end{itemize}
  \end{block}
  
\end{frame}

\begin{frame}\frametitle{Configurando... (3)}

 \begin{block}{Generación de claves (Windows)}
  \begin{enumerate}
   \item Usando putty key generator generar una clave. Bajar el binario e instalarlo. \pause 
    \item Reemplazar en Key comment el tu\_usuario@despegar.com. \pause 
    \item Guardar las constraseñas generadas. (tanto la pública como la privada) \pause 
    \item Copiar la clave que se generó (Public key for pasting into OpenSHH...). \pause 
    \item Ir a \url{https://github.com/settings/keys} y pegar la clave generada con putty key generator.
  \end{enumerate}

  \end{block}
\end{frame}

\section{Uso de Git}

\subsection{Flujo de Git}

\begin{frame}\frametitle{Flujo de Git}
  \begin{figure}
    \includegraphics[scale=1]{flujo-git.png} 
    \caption{Ciclo de vida de un archivo}
  \end{figure}
\end{frame}

\begin{frame}\frametitle{Flujo de Git (2)}
  
  \begin{block}{Referencia}
    \begin{itemize}
    \item \alert{Untracked:} Archivos nuevos (similar a SVN). \pause
    \item \alert{Unmodified:} Archivos sin cambiar desde el estado HEAD de nuestro repositorio local. \pause
    \item \alert{Modified:} Archivos modificados, aún sin stage. \pause
    \item \alert{Staged:} Archivos listos para ser commiteados.
    \end{itemize}
  \end{block}
  
\end{frame}

\begin{frame}{Non-fast-forward updates}
  \begin{block}{¿Repositorio local? }
    Algo para resaltar es que, a diferencia de p. ej. SVN, lo que nosotros mandamos (\textbf{push}) al servidor no son sólo cambios, sino
    el estado completo del repositorio. Es por eso que si intentamos subir nuestros cambios mientras que otro modificó archivos,
    por más de que sean diferentes, el servidor lo va a rechazar. Nos avisa que estaríamos perdiendo cambios y nos obliga
    a hacer un \textbf{pull} para que los mezcle. Esto se conoce como non-fast-forward updates.
  \end{block}
\end{frame}

\subsection{Tiremos unos comandos}

\begin{frame}[fragile]{Comandos}
  \begin{block}{Inicializar repositorio}
    \begin{verbatim}
  $ echo "#test" >> README.md
  $ git init
  $ git add README.md
  $ git commit -m "First commit"
  $ git remote add \
  		origin git@github.com:<user>/<my-repo>.git
  $ git push -u origin master
    \end{verbatim}
  \end{block} \pause
  
  \begin{block}{Clonar repositorio remoto (configura automáticamente el orígen)}
    \begin{verbatim}
  $ git clone git@github.com:<user>/<my-repo>.git
    \end{verbatim}
  \end{block}
  
\end{frame}

\begin{frame}[fragile]{Comandos (2)}
  \begin{block}{Crear ramas}
    \begin{verbatim}
  $ git branch <RAMA> && git checkout <RAMA>
    # Crear la rama en el punto actual.                 
    # Es necesario hacer checkout a la misma.
  $ git checkout -b <RAMA>  
    # Crear rama en el punto actual y hacerle checkout.
  $ echo "# <nombre>" >> <nombre>.md
    # Crear un nuevo archivo.
  $ git add <nombre>.md
    # Agregamos al registro.
  $ git status
    # Verificamos el estado del repo local.
    \end{verbatim}
  \end{block}
\end{frame}

\begin{frame}[fragile]{Comandos (3)}
  \begin{block}{Commit sobre la rama}
    \begin{verbatim}
  $ git commit -m "Agregar buen mensaje"
    # Realizamos el commit de los cambios.
  $ git log
    # Ver los cambios realizados.
    \end{verbatim}
  \end{block} \pause
  
  \begin{block}{Subiendo la rama local}
    \begin{verbatim}
$ git push origin <RAMA>
  # Subimos el estado de nuestro repo al repo remoto
$ git checkout -b merging && git merge <RAMA>
  # Vamos a la rama merging, luego hacemos un 
  merge de la rama, Las diferencias se resuelven 
  automáticamente si es posible. En caso de conflictos,
  el proceso se detiene (merging) es necesario manual.
  \end{verbatim}
  \end{block}
\end{frame}

\subsection{Conflictos... conflictos everywhere}

\begin{frame}[fragile]{Conflictos... conflictos everywhere}

  \begin{block}{Resolver conflictos}
    \begin{verbatim}
$ git push origin merging
  # Intentamos subir todos los cambios...
$ git pull && git status
  # Traemos y vemos la situación actual del merge
$ git mergetool
  # Muestra los conflictos y las diferencias
  # Si es que configuramos un mergetool...
    \end{verbatim}
  \end{block}
  \begin{figure}
    \includegraphics[scale=0.15]{everywhere.jpg} 
  \end{figure}
\end{frame}

\subsection{Conflictos con mergetool}

\begin{frame}[fragile]{Merge tool}

  \begin{block}{Configuración}
      \begin{verbatim}
  $ vim ~/.gitconfig
  [merge]
    tool = meldMerge
  [mergetool "meldMerge"]
    cmd = meld --diff $LOCAL $MERGED $REMOTE
  [diff]
    tool = meldDiff
  [difftool "meldDiff"]
    cmd = meld --diff $LOCAL $REMOTE
      \end{verbatim}
  \end{block}

\end{frame}

\begin{frame}[fragile]{Merge tool (2)}

  \begin{block}{Ejecución}
      \begin{verbatim}
  $ git difftool
  $ git mergetool
      \end{verbatim}
  \end{block} \pause

  \begin{block}{Herramientas}
    Algunas opciones:
      \begin{itemize}
       \item Meld
       \item Beyond Compare
      \end{itemize}\pause
	Mas completas:
      \begin{itemize}
       \item gitk
       \item GitKraken
      \end{itemize}	
  \end{block}
  
\end{frame}

\section{Branching}

\subsection{Uso de Branches}

\begin{frame}[fragile]\frametitle{Creando y obteniéndose ramas}

    \begin{block}{Sujeto ``A`` empieza una feature}
      Cuando creamos una rama, se desprende de aquella en la que estamos parados. (\$ git status)
      \begin{verbatim}
$ git checkout -b feature-nueva
$ git push origin feature-nueva
      \end{verbatim}
    \end{block} \pause
    
    \begin{block}{Sujeto ''B'' quiere trabajar en la misma feature}
      \begin{verbatim}
$ git fetch  origin
$ git branch -a
# *master
# remotes/origin/HEAD -> origin/master
# remotes/origin/feature-nueva
$ git checkout feature-nueva
      \end{verbatim}
    \end{block}
  
\end{frame}

\begin{frame}[fragile]\frametitle{Borrar una rama}

    \begin{block}{En mi repo local}
      Si es una rama local o sólo queremos eliminarla de nuestro repo, este paso es único.
      \begin{verbatim}
$ git branch -d <rama>
      \end{verbatim}
    \end{block} \pause
    
    \begin{block}{En origen remoto}
      Si además queremos borrarla del origen remoto
      \begin{verbatim}
$ git push origin :<rama>
      \end{verbatim}
    \end{block}
  
\end{frame}

\subsection{Buenas practicas}

%   \begin{figure}
%    \includegraphics[height=170px, width=260px]{branch-model.png} 
%	\url{ http://nvie.com/posts/a-successful-git-branching-model/}
%  \end{figure}

\begin{frame}{Main Branches}
\begin{figure}
	\begin{minipage}{0.3\textwidth}
		\includegraphics[scale=0.20,left]{main-branches.png}
	\end{minipage}
\begin{minipage}{0.6\textwidth}\raggedright
\begin{itemize}
  \item Se conocen como nuestras ramas maestras.
  \item Master es siempre nuestra rama productiva.
  \item Develop es la rama de integración, donde podemos trabajar tranquilos. Sabemos que puede no estar estable.
  \item En un equipo grande quizás nos falte algo mas.
\end{itemize}	
\end{minipage}
\end{figure}
\end{frame}

\begin{frame}{Support Branches - Featured branches}
\begin{figure}
	\begin{minipage}{0.3\textwidth}
		\includegraphics[scale=0.20,left]{fb.png}
	\end{minipage}
\begin{minipage}{0.6\textwidth}\raggedright
\begin{itemize}
  \item Se conocen como ramas de soporte.
  \item Nos permite desarrollar nuevos cambios sin afectar la rama develop.
  \item La creamos: git checkout -b myfeature develop.
  \item Luego de terminar volvemos a develop: git checkout develop.
  \item Integramos la rama feature en develop:  git merge --no-ff myfeature
\end{itemize}
\end{minipage}
\end{figure}
\end{frame}

\begin{frame}{Importancia de usar --no-ff}
\begin{figure}
	\begin{minipage}{0.35\textwidth}
		\includegraphics[scale=0.15,left]{merge-without-ff.png}
	\end{minipage}
\begin{minipage}{0.6\textwidth}\raggedright
\begin{itemize}
  \item El flag --no-ff, nos permite integrar los cambios en un nuevo commit.
  \item Sin el flag no podríamos identificar fácilmente los cambios del feature.
  \item En caso de necesitar volver atrás los cambios mergeados es mas fácil.
\end{itemize}
\end{minipage}
\end{figure}
\end{frame}

\begin{frame}{Support Branches - Release branches}
\begin{figure}
	\begin{minipage}{0.3\textwidth}
		\includegraphics[scale=0.50,left]{rb.jpg}
	\end{minipage}
\begin{minipage}{0.6\textwidth}\raggedright
\begin{itemize}
  \item Release branches nos permiten preparar un despliegue en producción: git checkout -b release-1.0 develop
  \item Solo deberíamos realizara pequeños bugfix.
  \item Cada pequeño bugfix podemos integrarlo en develop.
  \item Utilizando github se realiza un pull request.
  \item Sin github es podria hacerse en merge en master (usar --no-ff)
\end{itemize}
\end{minipage}
\end{figure}
\end{frame}

\begin{frame}{Support Branches - Hotfix branches}
\begin{figure}
	\begin{minipage}{0.3\textwidth}
		\includegraphics[scale=0.15,left]{hotfix-branches.png}
	\end{minipage}
\begin{minipage}{0.6\textwidth}\raggedright
\begin{itemize}
  \item Murio prod!!! rajemo!!!.
  \item No, usamos una hotfix branch.
  \item La creamos: git checkout -b hotfix-1.2.1 master
  \item Arreglamos el problema podemos subir a master.
  \item git checkout master \&\& git merge --no-ff hotfix-1.2.1
  \item No olvidarse de hacer el tag: git tag -a 1.2.1 -m "hotfix de prod"
\end{itemize}
\end{minipage}
\end{figure}
\end{frame}

\begin{frame}[fragile]\frametitle{Sino hay tabla}
	\begin{figure}
		\begin{minipage}{0.3\textwidth}
			\includegraphics[scale=0.15,left]{80a.png}
		\end{minipage}
	\begin{minipage}{0.68\textwidth}\raggedright
    \begin{itemize}
      \item Commit sin comentarios, hay tabla.
	  \item Features en el master, hay tabla.
	  \item Hotfix en el master, hay tabla.	  
	  \item "git add ." con conflictos de merge, hay tabla.
	  \item git merge sin usar flag --no-ff, hay tabla.
	  \item merge directo sobre el master sin pull request, hay tabla.
	  \item git reset sobre commit publico, hay tabla.
	  \item commit --amed sobre commit publico, hay tabla.
	  \item rebase sobre rama publica, hay tabla.
    \end{itemize}	
    \end{minipage}
	\end{figure}
\end{frame}

\subsection{Tags}

\begin{frame}[fragile]{Tagging}
  \begin{block}{Listar}
    \begin{verbatim}
  $ git tag
    \end{verbatim}
  \end{block} 

  \begin{block}{Crear un tag}
    \begin{verbatim}
  $ git tag -a v1.4 -m "creando el tag 1.4"
    \end{verbatim}
  \end{block}

  \begin{block}{Eliminarlo}
    \begin{verbatim}
  $ git push origin :refs/tags/v1.4
  #Evitar eliminar un tag productivo.
    \end{verbatim}
  \end{block}
  \footnotetext{Para más información: \url{http://git-scm.com/book/en/Git-Basics-Tagging}}
\end{frame}


\section{Adicionales}

\subsection{Tips: Stash, Cherry, Cherry pick}

\begin{frame}{Stash}
  \begin{block}{Stash}
  ¿Qué pasa si estoy por la mitad de un cambio y necesito moverme de branch para ver otra cosa? Si intentamos
  hacer checkout, nos va a pedir que commiteemos los cambios ya que de otra manera los vamos a perder. Otra opción
  es el comando \textbf{stash}. \footnotemark \\
  Básicamente, la idea es que los cambios quedan almacenados y al hacer git status no los vemos. Una vez que volvamos
  a trabajar en ese branch, podemos recuperarlos con git stash apply.
  \end{block}
  
  \footnotetext{Para más información: \url{http://git-scm.com/book/en/Git-Tools-Stashing}}
\end{frame}

\begin{frame}{Cherry - Cherry Pick}
  \begin{block}{Cherry}
  En ciertas circunstancias podríamos llegar a necesitar aplicar cambios en ramas que divergieron. Es decir, si 
  las mergeara, traería otros cambios que yo no quiero. El comando cherry nos permite ver, dado un branch, qué cambios
  se hicieron desde que las dos ramas se separaron. \url{http://git-scm.com/docs/git-cherry}.
  \end{block}
  \pause
  \begin{block}{Cherry Pick}
  Cherry pick nos permite aplicar esos cambios de uno o más commits específicos a un branch. \url{http://git-scm.com/docs/git-cherry-pick}.
  \end{block}
\end{frame}

\subsection{Gitignore}

\begin{frame}{Gitignore}
  \begin{block}{.gitignore}
      Tenemos la posibilidad de decirle a git que ignore cierto patrón de archivos en el índice. Para ello basta crear un archivo con el nombre .gitignore
      y escribir adentro los nombres de archivos o carpetas que queremos ignorar. Si queremos que aplique a todo el repositorio, lo ponemos en la raíz.
      Si es más específico pueden agregarse múltiples en diferentes carpetas. Algo importante es nunca poner .gitignore adentro ya que el .gitignore en
      sí mismo debe ser commiteado para que afecte a todos los que utilizan el repositorio.
  \end{block}

\end{frame}

\subsection{Referencias}

\begin{frame}{Referencias}
 \begin{block}{Links útiles}
    \begin{itemize}
     \item Git official website, \url{http://git-scm.com/}
     \item Pro Git Book, \url{http://git-scm.com/book}
     \item Git Tutorial, \url{http://www.slideshare.net/eykanal/git-introductory-talk}
     \item Git guía rápida, \url{http://www.edy.es/dev/docs/git-guia-rapida/}
     \item Basic branching and merging, \url{http://git-scm.com/book/en/Git-Branching-Basic-Branching-and-Merging}
     \item A successful branch model, \url{http://nvie.com/posts/a-successful-git-branching-model/}
    \end{itemize}

 \end{block}
\end{frame}

\begin{frame}{Referencias (2)}
 \begin{block}{Links útiles}
    \begin{itemize}
          \item Some git tips, \url{http://mislav.uniqpath.com/2010/07/git-tips/}
     \item Revert commit by hash, \url{http://stackoverflow.com/questions/1895059/git-revert-to-a-commit-by-sha-hash}
     \item Revert bad merge, \url{https://condor-wiki.cs.wisc.edu/index.cgi/wiki?p=RevertingBadMerges}
    \end{itemize}

 \end{block}
\end{frame}


\end{document}
